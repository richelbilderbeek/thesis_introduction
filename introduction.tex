\chapter{Introduction}
\label{chapter_introduction}
%% Start the actual chapter on a new page.
\newpage

\noindent

Once upon a time, there was the first living organism, the 
First Universal Common Ancestor (FUCA).
We do not know when it lived.

% 
%    ----- FUCA
%
% ---+---- (time)
%    t0

One unknown day, FUCA speciated, resulting in two species.
This event doubled the biodiversity on Earth.
The two species, which we will call species A and B
are sister species.
We do not know what caused the speciation.

%     +---- A
%  ---+
%     +---- B
%
% -+--+--- (time)
%  t0 t1

Both species A had their unknown histories: they speciated themselves,
and they and/or their descendants went extinct. 
Extinction is a common event. Let's assume A and/or its clade went
exctint and that species B created a sister species C. Species B
and C will give rise to all biodiversity. This ancestor of species B and C 
is called the Last Universal Common Ancestor and lived around [then].

            
%     +--------x   A
%  ---+     
%     |     +------- B
%     +-----+
%           +------- C
%
% -+--+-----+--+---- (time)
%  t0 t1    t2 t3

The biodiversity derived from LUCA is important to us humans, apart from
that is has created us. A review paper (Cardinale et al.) shows that 
biodiversity unsually improves ecosystem services that are 
positive for human well-being.

%
% Ecosystem function
%
%   |  _______
%   | /
%   |/
%   +--------------
%
%     Biological diversity
%

Speciation is the process that increases biological diversity 
and is therefore worth to study.
We can look at the mechanism behind each speciation event,
but we can also look at patterns 
of speciation events through evolutionary time.

%
% [picture of speciation machanism] | [picture of a trait that increases speciation rate]
%                                   | 
% Speciation mechanism:             | Speciation events though time:                           
% what caused this speciation       | what are the patterns of
% event to occur?                   | speciation events?
% In this case, a geographics       | In this case, a trait that was a key innovation
% barrier caused the initial        | gave rise to a higher speciation rate
% species to diverge                | 
%

The field of phylogenetics is the research discipline that
infers phylogenies from (mostly) genetic data and studies the 
branching patterns of phylogenies found in nature.
One example is the Tree Of Life, which uses aligned protein sequences
of many species to create one big phylogeny of all (sequenced) life on Earth

%
% [Tree Of Life]
%
% The Tree Of Life [reference]

To create a phylogeny from a protein alignment, there have been many assumptions 
made, regarding the evolution of a protein sequence, 
the rate(s) at which this happens and the rate(s) at which 
a branching/speciation event takes place. 







